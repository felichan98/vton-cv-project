\begin{multicols}{2}
\tableofcontents
\section{Introduction}

\subsection{Problem statement}

Our project will be focusing on designing a 2D image-based virtual try on system. Virtual try-on consists in generating an image of a reference person wearing a given try-on garment. 

\subsection{Related works}
In order to solve this problem we have analyzed already existing literature and their approaches to the same problem.

The CP-VTON paper\cite{CP-VTON} identifies four main requirements that need to be accomplished by a virtual try-on system:
\begin{itemize}
\item warping the garment according to the body shape and pose of the target person; 
\item transferring the texture of the garment on the target person without losing important details; 
\item merging the image of the target person with the warped result in a plausible way; and 
\item render light and shades of the final image correctly, to ensure realism.
\end{itemize}


One of the most important advancements made by the CP-VTON architecture is the introduction of a warping module that computes a learnable Thin-Plate Spline transformation that warps the garment in a reliable way.

We obtained access to the Dress-Code dataset \cite{dress-code}, collected by AImageLab, which, given its size, may potentially boost the efficacy of our system.

We also noticed that, in recent years, that there have been a lot of improvements to the quality of the generated images through the introduction of transformer-based modules. 
This approach is followed by the paper \textbf{Dual-Branch Collaborative Transformer for Virtual Try-On}\cite{dual-branch} to solve the virtual try-on problem with great results, as such we will also attempt to implement a similar transformer-based architecture. 

\subsection{General approach}
Our system will be subdivided into different modules each one designed to solve specific step of the process.
\begin{itemize}
\item pre-processing: this module handles all which regards the image enhancement (denoising, light adjustement, etc...), performs the background removal task;

\item person representation: this module performs pose estimation and the semantic segmentation of the person into their body parts;

\item warping module: this module implements a geometric transformation that envelops the body shape of the subject with the fabric of the clothing item;

\item try-on: as the last module in the pipeline, this part generates a new image by composing the warped garment over the subject and should ensure the satisfaction of the requirements stated in the above section.
\end{itemize}


\end{multicols}